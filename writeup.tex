%UsePlainTex
\font \rm cmr10 \magnification \magstep 2 \rm 
\input tikz.tex

In our investigations of fitness landscapes, we considered a
three loci system.  Each locus had two states or
``alleles'', which we denoted with the binary digits `0' or
`1'.  Using this notation, we found it convenient to
represent genotypes as binary numbers between 0 and 7.  A
fitness landscape is a function from bitstrings to real
numbers.  

Linkage is not considered in our investigations, so the
order in which the loci are written is not relevant.
Furthermore, whether we call a state `0' or `1' is also not
relevant.  Perhaps somewhat surprisingly, these
``symmetries'' correspond exactly to the symmetric
transformations of the three dimensional cube, including
those which flip the cube outside of three dimensional space
and back in with a different orientation.  This equivalence
is realized by placing a cube in a Cartesian space with
vertices at
(0,0,0),(1,0,0),(0,1,0),(0,0,1),(1,1,0),(1,0,1),(0,1,1), and
(1,1,1).  The bit strings correspond exactly to the vertices
by the mapping $$b_1b_2b_3 \rightarrow (b_1,b_2,b_3).$$

Our consideration of fitness landscapes was focussed on the
{\it order} of the fitness values.  For this reason, we are
able to push the analogy with the cube an extra step, and
visual a fitness landscape as a cube with directed edges,
where the direction is determined by comparison of the
fitness values of the two connected vertices.  (Our
convention was that the arrows pointed toward the higher
fitness, so we can think of the direction as the tendency of
evolution as well.)

We desired a complete listing of all three loci two state
fitness landscapes with satisfy the Dyck condition described
in the main text.  To avoid redundancy, we did not include
any two fitness landscapes where, when viewed as cubes with
directed edges, one was a symmetric transformation of the
other.  For example, the following three fitness landscapes
are symmetric transformations of each other.  

\vskip \baselineskip

\def\bob#1#2#3{\count255=#3 \advance \count255 by
1 \ifnum \count255=2 \advance\count255 by -2 #2#1\number\count255\else#2#1\number\count255\fi}

\def\sue#1#2#3{\count255=#3\advance\count255 by 1 \ifnum
\count255=2 \advance\count255 by -2
#1#2\number\count255\else#1#2\number\count255\fi}

\tikzpicture[nodes={scale=0.5},scale=0.5] {\node (000) at (0,0) {000};
		\node (001) at (0,2) {001};
		\node (010) at (2,0) {010};
		\node (100) at (1,1) {100};
		\node (110) at (3,1) {110};
		\node (011) at (2,2) {011};
		\node (101) at (1,3) {101};
		\node (111) at (3,3) {111};
		\draw (000) edge [->] (001);
		\draw (000) edge [->](010);
		\draw (000) edge [->](100);
		\draw (100) edge [->](110);
		\draw (010) edge [->](110);
		\draw (001) edge [<-](011);
		\draw (010) edge [<-](011);
		\draw (001) edge [<-](101);
		\draw (101) edge [<-](111);
		\draw (011) edge [->](111);
		\draw (101) edge [->](100);
		\draw (111) edge [->](110);
		};
\endtikzpicture
\quad 
\tikzpicture[nodes={scale=0.5},scale=0.5] 
		{\node (000) at (0,0) {\bob000};
		\node (001) at (0,2) {\bob001};
		\node (010) at (2,0) {\bob010};
		\node (100) at (1,1) {\bob100};
		\node (110) at (3,1) {\bob110};
		\node (011) at (2,2) {\bob011};
		\node (101) at (1,3) {\bob101};
		\node (111) at (3,3) {\bob111};
		\draw (000) edge [->] (001);
		\draw (000) edge [->](010);
		\draw (000) edge [->](100);
		\draw (100) edge [->](110);
		\draw (010) edge [->](110);
		\draw (001) edge [<-](011);
		\draw (010) edge [<-](011);
		\draw (001) edge [<-](101);
		\draw (101) edge [<-](111);
		\draw (011) edge [->](111);
		\draw (101) edge [->](100);
		\draw (111) edge [->](110);
		};
\endtikzpicture{
\quad
\tikzpicture[nodes={scale=0.5},scale=0.5] 
		{\node (000) at (0,0) {\sue000};
		\node (001) at (0,2) {\sue001};
		\node (010) at (2,0) {\sue010};
		\node (100) at (1,1) {\sue100};
		\node (110) at (3,1) {\sue110};
		\node (011) at (2,2) {\sue011};
		\node (101) at (1,3) {\sue101};
		\node (111) at (3,3) {\sue111};
		\draw (000) edge [->] (001);
		\draw (000) edge [->](010);
		\draw (000) edge [->](100);
		\draw (100) edge [->](110);
		\draw (010) edge [->](110);
		\draw (001) edge [<-](011);
		\draw (010) edge [<-](011);
		\draw (001) edge [<-](101);
		\draw (101) edge [<-](111);
		\draw (011) edge [->](111);
		\draw (101) edge [->](100);
		\draw (111) edge [->](110);
		};
\endtikzpicture

\vskip \baselineskip

In the end we
found 20 ``unique'' fitness landscapes which satisfied the
Dyck condition.  This was out of 54 possible ``unique''
fitness landscapes in total.  

\bye



